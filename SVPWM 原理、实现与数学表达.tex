\documentclass{ctexart}
\usepackage{amssymb,amsfonts,amsmath,amsthm}
%--------
% 设置字体
%--------




%----------
% 版面设置
%----------
%首段缩进
\usepackage{indentfirst}
\setlength{\parindent}{2em}

%行距
\renewcommand{\baselinestretch}{1.5} % 1.5倍行距

%页边距
\usepackage[a4paper]{geometry}
\geometry{verbose,
  tmargin=2cm,% 上边距
  bmargin=2cm,% 下边距
  lmargin=3cm,% 左边距
  rmargin=3cm % 右边距
}


%----------
% 其他宏包
%----------
\usepackage{enumerate}
\usepackage{geometry}
\usepackage{fontspec}
\usepackage{listings}
\usepackage{xcolor}
\usepackage{graphicx}
\usepackage{subfigure}
\usepackage{caption}
\usepackage{pdfpages}%插入PDF

\graphicspath{{figures/}{logo/}}%声明图片和logo目录

%-----------
%页眉页脚
%-----------
\usepackage{fancyhdr}
\setlength{\headheight}{14pt}
\pagestyle{fancy}
\lhead{}
\chead{\kaishu SVPWM原理、实现与数学表达}
\rhead{ }
\cfoot{\thepage}


%-----------
%取消Section编号
%-----------
%\setcounter{secnumdepth}{-2}


%-------------
%公式编号加入节section
%-------------
\numberwithin{equation}{section}


%-------------
%new command 公式
%-------------



%-----------
%正文
%-----------


\begin{document}


\title{\heiti SVPWM原理、实现与数学表达}    
\author{\kaishu 李华康 \\huakang.li@outlook.com}  
\date{2020.06}
\maketitle


SVPWM的主要思想是以三相对称正弦波电压供电时三相对称电动机定子理想磁链圆为参考标准,
以三相逆变器不同开关模式作适当的切换,从而形成PWM波,以所形成的实际磁链矢量来追踪其准确磁链圆。
——百度百科(王兆安《电力电子技术基础》)。

电压空间矢量调制SVM的思想源于交流异步电机变频调速。——陈坚《电力电子学》。

作为电气专业的学生,我们很多时候是基于电机来理解空间矢量的。但是,给它过多的物理含义其实并没有意义。空间矢量调制完全可以通过数学的方法来理解。

\section{相量、矢量、时间、空间}
在正式讨论SVPWM之前,我们要明确与辨析一些基本概念的含义,即相量,矢量,时间与空间。

\subsection{相量}
需要说明的是,相量对于分析SVPWM并没有太多帮助,但是明确的区分相量和空间矢量。相量是为了分析正弦稳态电路而人为定义的一个矢量。对于一个正弦稳态的电气信号,它包含三个基本要素,即幅值、频率和初相位。下面以电压信号为例,三相对称电压信号的表达式为

\begin{equation}
  \begin{aligned}
  	&u_A(t) = U_m \sin (\omega t)\\
  	&u_B(t) = U_m \sin (\omega t - 2/3\pi)\\
  	&u_C(t) = U_m \sin (\omega t + 2/3\pi)
  \end{aligned}
\end{equation}
$ u_A $、$ u_B $、$ u_C $的幅值均为$ U_m $,频率均为$ \omega $,而初相位分别为0,-2/3$ \pi $和2/3$ \pi $。所以要区分$ u_A $、$ u_B $、$ u_C $,只需要将它们的初相位单独标识出来即可,也就是说,$ u_A $、$ u_B $、$ u_C $可分别记为$ \angle 0 $、$ \angle -2/3\pi $、$ 2/3\pi $。在同一个电路系统中,往往只有一个频率,但是可能有几个电压,所以在初相位信息的基础上再加上幅值信息,就可以表达出这个系统中的所有电压电流信号了,而这个表示系统就是相量表达。因此$ u_A $、$ u_B $、$ u_C $完整的相量表达就是$ \dot U_A = U_m\angle 0 $、$ \dot U_B = U_m\angle -2/3\pi $、$\dot U_C = U_m2/3\pi $。

但是光表示还是远远不够的,这套表示系统还需要具备可运算的功能才有实际用处。可以很清楚的看到,相量之间的运算其实就是正弦函数之间的运算。正弦函数直接进行运算往往非常麻烦,但是如果能将正弦函数之间的运算转换到更加简单的系统当中去,相量表示法将变得更加有价值。而欧拉公式(式1.2)正是这样一个非常好的连接工具。
\begin{equation}
  Ue^{j(\omega t + \varphi)} = U\cos (\omega t + \varphi) +jU\sin (\omega t + \varphi)
\end{equation}
一个复指数函数的实部或虚部都包含了正弦信号,且稳态正弦信号中角频率固定不变,$Ue^{j\varphi} \to U\sin(wt+\varphi)$ 这个映射保持加法运算,是加法群同构\footnote{何子涵同学指导},因此通过对复指数函数的运算就可以实现对相量的运算。所以$ u_A $、$ u_B $、$ u_C $用带指数函数的相量形式可以写作$ \dot U_A = U_m e^{j0} $、$ \dot U_B = U_m e^{j-2/3\pi} $、$ \dot U_C = U_m e^{j2/3\pi} $。

\subsection{矢量}
又称相量,是一种既有大小又有方向的量。从相量的复指数函数定义可以看出,相量其实就是一种矢量,其大小可以表示电信号的幅值,其方向可以表示电信号的相位信息,并且它的运算遵循平行四边形法则。可以看到相量代表的矢量,有着明确的物理意义。但是SVPWM中的矢量则很大程度上是为了数学上的方便而认为造出来的。

\subsection{空间}
事实上任何矢量都必须存在于某一个空间,相量矢量也是,它的矢量空间就是复平面。所以单说空间矢量其实是意义不明的,在SVPWM中的空间矢量,是人为规定的在某一个特定空间内存在的特定矢量。在叙述这个特定矢量空间之前,让我们先从另一个矢量空间说起,如图\ref{abc-space}左侧所示,并将其记为“ABC空间”。在这个空间中,有ABC三个轴,这三个轴上的基底长度都是1,方向依次增加120$ ^\circ $ 。
同时有幅值为1的三相对称正弦信号$u_a$、$u_b$、$u_c$,相位依次滞后120$ ^\circ $,如图\ref{abc-space}右侧所示,并将其记为“abc系统”。

将abc系统中$u_a$、$u_b$、$u_c$的大小分别作为ABC空间中A、B、C三个轴上矢量$\vec{x_{a}}$、$\vec{x_{b}}$、$\vec{x_{c}}$的长度大小,并将这三个矢量依据平行四边形法则进行合成得到矢量$\vec{x_F}$。
比如在$ \omega t = \pi /2 $时,$ \vert \vec{x_a} \vert = 1 $,$ \vert \vec{x_b} \vert = 1/2 $,$ \vert \vec{x_c} \vert = 1/2 $,合成相量$  \vec{x_F} $的长度为3/2,而方向与A轴基底方向相同。
类似的,在$ \omega t = \frac{7}{6}\pi $时,$ \vert \vec{x_a} \vert = 1/2 $,$ \vert \vec{x_b} \vert = 1 $,$ \vert \vec{x_c} \vert = 1/2 $,合成相量$  \vec{x_F} $的长度为3/2,而方向与B轴基底方向相同。
在$ \omega t = \frac{11}{6}\pi $时,$ \vert \vec{x_a} \vert = 1/2 $,$ \vert \vec{x_b} \vert = 1/2 $,$ \vert \vec{x_c} \vert = 1 $,合成相量$  \vec{x_F} $的长度为3/2,而方向与C轴基底方向相同。可以发现,合成矢量 $ \vec{x_F} $的长度并没有改变而只是方向发生了变化。事实上,通过数学可以证明,合成矢量 $ \vec{x_F} $是一个长度保持3/2不变且在空间中匀速旋转的矢量,旋转方向为逆时针方向,旋转速度等于abc系统中角频率 $ \omega $。

\begin{figure}[hbtp]
\centering
\begin{minipage}[b]{0.48\linewidth}
\centering
\includegraphics[width = \linewidth]{ABC-vector-space.pdf}
\caption{ABC空间 - abc系统}
\label{abc-space}
\end{minipage}
\hspace{10pt}
\begin{minipage}[b]{0.48\linewidth}
\centering
\includegraphics[width = \linewidth]{alpha-beta-space.pdf}
\caption{$\alpha \beta $空间 - $\alpha \beta $系统}
\label{alpha-beta-space}
\end{minipage}
\end{figure}

由以上分析可知,ABC空间中的$ \vec{x_F} $与abc系统二者之间可以通过一套数学公式相互转换。事实上,要合成矢量$ \vec{x_F} $并不需要三个基底,而只需要两个不平行的基底就行。通常情况,可以选择两个正交的基底,如图\ref{alpha-beta-space}左侧所示,为了与之前的三个基底的表达方式进行区别,将其记为“$\alpha \beta$空间”。$\alpha \beta$空间中共有两个轴,基底长度均为3/2,方向上$ \beta $轴比$ \alpha $轴大90$ ^\circ $。设在$\alpha \beta$空间中有与ABC空间中相同的旋转矢量$ \vec{x_F} $,并将这个矢量在$ \alpha $轴和$ \beta $轴上分解为$\vec{x_{\alpha}}$和$\vec{x_{\beta}}$。将$\vec{x_{\alpha}}$和$\vec{x_{\beta}}$的长度大小记为${u_{\alpha}}$和${u_{\beta}}$,画出${u_{\alpha}}$和${u_{\beta}}$随旋转相位$\omega t$的变化关系如图\ref{alpha-beta-space}右侧所示,其中让$\alpha \beta$空间中旋转矢量的初相位与$ABC$空间中旋转矢量的初相位保持一致。将${u_{\alpha}}$和${u_{\beta}}$这样的正交两相正弦信号记为“$\alpha \beta$系统”。这样abc系统就通过空间旋转矢量与$\alpha \beta$系统建立了联系,可以说这两个系统其实表达了相同的信息。为什么一个三相信号可以转换为一个两相信号了,这是因为对称的三相信号彼此并不完全独立,它们在任何时候的和都为0,因此只需要两相信号就可以完整地表达三相信息。

abc系统向$\alpha \beta$系统的变换就是所谓的Clarke变换,见式\ref{clarke transformation},相应地,$\alpha \beta$系统向abc系统的变换就是Clarke反变换,见式\ref{inverse clarke transformation}。在上面的分析中我们让旋转矢量的长度相等,得到${u_{\alpha}}$和${u_{\beta}}$的幅值是3/2,通过调整Clarke变换与反变换中的系数$m$和$m'$,可以调整abc系统和$\alpha \beta$系统中的幅值大小。$m=2/3$和$m'=1$时,abc系统和$\alpha \beta$系统中的幅值大小相等;$m=\sqrt{2/3}$和$m'=\sqrt{2/3}$时,abc系统和$\alpha \beta$系统中的功率大小相等。
\begin{align}
  \text{Clarke Transformation:}
  &\begin{bmatrix}
    u_{\alpha}\\
    u_{\beta}
  \end{bmatrix}
  =
  m
  \begin{bmatrix}
    1 & -\frac{1}{2} & -\frac{1}{2}\\
    0 & \frac{\sqrt{3}}{2} & -\frac{\sqrt{3}}{2}
  \end{bmatrix}
  \begin{bmatrix}
    u_a\\
    u_b\\
    u_c
  \end{bmatrix}
  \label{clarke transformation}
  \\
  \text{Inverse Clarke Transformation:}
  &\begin{bmatrix}
    u_a\\
    u_b\\
    u_c
  \end{bmatrix}
  =
  m'
  \begin{bmatrix}
    1 & 0\\
    -\frac{1}{2} & \frac{\sqrt{3}}{2}\\
    -\frac{1}{2} & -\frac{\sqrt{3}}{2}
  \end{bmatrix}
  \begin{bmatrix}
    u_{\alpha}\\
    u_{\beta}
  \end{bmatrix}
  \label{inverse clarke transformation}
\end{align}

在之前我们借助旋转矢量讲述了这个变换,其实这个旋转矢量在Clarke变换的数学解释中完全是多余的,但是空间旋转矢量对于理解与应用Clarke变换非常有帮助。

\subsection{时间}
在对正弦稳态信号的分析中,时间信息其实是和频率信息高度捆绑在一起的,在相量系统中,时间信号伴随着频率系统一起被省略。而在空间矢量系统中,频率和时间则可以被拿出来讨论,但是这里的频率并不是正弦稳态信号中的频率,而是空间矢量旋转的角速度。严格来说,只有稳定的正弦信号才能有频率这一概念,如果旋转矢量的角速度发生变化,那么它不再对应一个稳定的三相对称信号,但是在瞬时,我们可以近似认为旋转矢量角速度不变,从而对应一个三相对称稳态正弦信号,虽然它在物理上并不存在,但是这将有助于我们跟踪和控制三相交流系统的频率。在这种定义下,频率可认为是电信号相位$\theta$对时间的导数,即
\begin{equation}
  \omega = \frac{d\theta}{dt}
\end{equation}

下面将给出一个实例具体阐述。如现在希望得到一个幅值和频率都从0开始随时间线性增加,直至0.2$s$后达到期待幅值100A和频率60Hz的三相交流电流。从0-0.2$s$的整体来看,这个三相系统都是不稳定的,频率和幅值也就无从谈起,但是如果我们认为它在瞬时“稳定”,那么问题就可以得到解决。因此
\begin{equation}
  \omega = \frac{d\theta}{dt} = 120\pi kt
\end{equation}
式中,$ k = 1/0.2 = 5 $。假定参考信号的初相位为0,则
\begin{equation}
  \theta = \int_0^{t} \omega dt = 60\pi kt^2
\end{equation}
因此三相交流电流的数学表达式为
\begin{equation}
  \begin{aligned}
  	&i_a = 100kt\sin (60\pi kt^2 )\\
  	&i_a = 100kt\sin (60\pi kt^2 -\frac{2}{3}\pi )\\
  	&i_a = 100kt\sin (60\pi kt^2 + \frac{2}{3}\pi )
  \end{aligned}
\end{equation}
0.2$ s $之后,各相电流维持为60Hz和100$ A $的正弦波,画出0-0.25$ s $三相电流波形,如图\ref{kw-kA-sine-wave}所示。

\begin{figure}[hbt]
  \centering
  \includegraphics[width = .8\linewidth ]{sin_wave_kA_kw}
  \caption{0-0.25$ s $三相电流信号波形}
  \label{kw-kA-sine-wave}
\end{figure}

\section{SVPWM原理}
SVPWM在最早出现时,与电机的旋转磁链有关,但是后来SVPWM也广泛用于其他三相电力电子变换和控制之中。这是因为任何三相对称正弦系统都能通过坐标变换的数学方式得到类似旋转磁链的空间旋转矢量。通过控制电机的磁链追踪理想磁链圆可以达到控制电机的目的,同样的,通过控制电力电子变换器的三相信号经坐标变换产生的空间矢量追踪理想矢量圆,也可以达到控制电力电子变换器的目的。

以三相两电平电压型逆变器为例。逆变器一共有3个桥臂,8种开关状态。

  















\end{document}